\chapter{INTRODUCTION}
\thispagestyle{empty}
\onehalfspacing
\pagestyle{fancy}
\fancyhf{}
\fancyhead[LE,RO]{\textit{\footnotesize \thepage}}
\fancyhead[RE,LO]{\textit{\footnotesize RFID Based Surveillance System for School Bus}}
%\fancyfoot[LE,LO]{\textit{\footnotesize Department of CSE}}
\fancyfoot[LE,RO]{\textit{\footnotesize Department of CSE}}
 
\renewcommand{\headrulewidth}{2pt}
\renewcommand{\footrulewidth}{1pt}
\section{Radio Frequency Identification(RFID)}
\par RFID (radio frequency identification) is a form of wireless communication that incorporates the use of electromagnetic or electrostatic coupling in the radio frequency portion of the electromagnetic spectrum to uniquely identify an object.
\par Every RFID system consists of three components: a scanning antenna, a transceiver and a transponder. When the scanning antenna and transceiver are combined, they are referred to as an RFID reader or interrogator. There are two types of RFID readers -- fixed readers and mobile readers. The RFID reader is a network-connected device that can be portable or permanently attached. It uses radio waves to transmit signals that activate the tag. Once activated, the tag sends a wave back to the antenna, where it is translated into data.
\par The transponder is in the RFID tag itself. The read range for RFID tags varies based on factors including the type of tag, type of reader, RFID frequency and interference in the surrounding environment or from other RFID tags and readers. Tags that have a stronger power source also have a longer read range.
\par RFID tags are made up of an integrated circuit (IC), an antenna and a substrate. The part of an RFID tag that encodes identifying information is called the RFID inlay.
\\
There are two main types of RFID tags:
\begin{itemize}
    \item Active RFID : An active RFID tag has its own power source, often a battery.
    \item Passive RFID : A passive RFID tag receives its power from the reading antenna, whose electromagnetic wave induces a current in the RFID tag's antenna.
\end{itemize}

\section{RFID Based Surveillance System}
A school bus system is one of the most important transportation methods for
students all over the world. But the potential safety risks associated with them
often go overlooked.
\par A school bus security system should be able to;
\begin{itemize}
	\item Monitor the vehicle and ensure students remain safe to and from  school.
	\item Capture the bus’s interior for comprehensive monitoring.
	\item RFID integration to track the entry of students.

\end{itemize}

\section{Problem Statement}
\par Waste can be separated between plastic and non-plastic via smart bins. Three machine learning models are utilised in this instance to classify the garbage.
\begin{itemize}
    \item Convolution Neural Network (CNN)
    \item Random Forest
    \item Gaussian Naive Bayes
\end{itemize}
\par By determining the values for precision, accuracy, recall, and f1-score, the three models are assessed. Therefore, locating the ideal waste classification model is simple.




























































































































































































